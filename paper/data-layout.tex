\section{Data layout}

    As already implicated, the data is divided into several buckets --like in a
    regular hashtable. 

    As to be expected, any element which may be part of the set is mapped onto
    a bucket injectively.
    The sets made up from the elements which would be mapped to a specific
    bucket $e(b)$ are disjunctive to the elements which would be mapped to
    another bucket of the same table:

    \begin{equation}
        e(b_1) \cap e(b_2) = \emptyset
    \end{equation}

    Note that $e(b)$ is the set of \em all\em{} elements which \em may\em{}
    potentially be part of the bucket, rather than elements which \em are\em{}
    elements of the bucket.

    \subsection{Number of buckets}

        So far, we're just talking about an average hashtable. However, we
        constrain the number of buckets $s$ to powers of two:

        \begin{equation}
            s \in \left\{ 2^n | n \in \mathbb{N} \right\}
        \end{equation}

        Assume having two hashtables $A$ and $B$, where the number of buckets of
        table $B$ is greater or equal than that of table $A$.
        The constraint ensures that, in this scenario, any bucket in $A$ is
        equivalent to a subset of the buckets of $B$, regarding the elements it
        may contain:

        \begin{equation}
            s(A) < s(B)
            \quad\Rightarrow\quad
            \forall_{0 \leq i \leq s(A)} : e(a_i) = \bigcup_{j\%s(B)=i} e(b_j)
        \end{equation}


    \subsection{Division into buckets}

        Naturally, a hashtable $H$ consists of a number of buckets $b_n$:

        \begin{equation}
            H = \left\{ b_n | n \in [0, s(H)) \right\}
        \end{equation}

        In regular hashtables, the bucket an element $i$ is put into is derived
        by applying the modulo on the hash-value $h$ calculated from the
        element:

        \begin{equation}
            n = h(i) \% s(H) \qquad i \in e(b_n)
        \end{equation}

        Taking into account the constraint described before, this would be
        equivalent with taking the least significant bits of the hash-value.
        However, we are using the \em most\em{} significant bits:

        \begin{equation}
            i \in e(b_n) \qquad n = h(i) \div \frac{2^N}{s}
        \end{equation}

        Where $N$ is the number of bits of a hash-value.
        
        Obviously, this preserves a subset of the natural numbers' ordering
        relation: using this mapping, hashes of items in later buckets are
        greater than those in earlier ones:

        \begin{equation}
            n_1 > n_2 \land i_1 \in e(b_{n_1}) \land i_2 \in e(b_{n_2})
            \Rightarrow i_1 > i_2
        \end{equation}

        This feature will proof especially useful when performing resizes and
        binary operations on sets with different bucket-counts.

    \subsection{Bucket sub-structure}

        Each bucket consists of an AVL tree on it's own, using the hashes as
        keys.
        The AVL tree consists of nodes, each containing one or more elements
        with identical hashes, e.g. a list of elements.


