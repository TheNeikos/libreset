\section{Data layout}

    As already implicated, the data is divided into several buckets --like in a
    regular hashtable. 

    As to be expected, any element which may be part of the set is mapped onto
    a bucket injectively.
    The sets made up from the elements which would be mapped to a specific
    bucket $e(b)$ are disjunctive to the elements which would be mapped to
    another bucket of the same table:

    \begin{equation}
        e(b_1) \cap e(b_2) = \emptyset
    \end{equation}

    Note that $e(b)$ is the set of \em all\em{} elements which \em may\em{}
    potentially be part of the bucket, rather than elements which \em are\em{}
    elements of the bucket.

