\section{Miscellaneous}
\label{sec:misc}

    This section serves the purpose of documenting miscellaneous algorithms used
    in the library.


    \subsection{Bit counting algorithm}
    \label{sec:misc-bitcount}

        A classical way of counting bits set in an integer is to perform shifts
        to the right, reading out the single bits and counting up a variable for
        each bit found set to $1$.
        This simple algorithm is in $O(n)$, where $n$ is the number of bits in
        an integer.
        This section will introduce an algorithm which performs the same task,
        but is in $O(log(n))$.

        An integer is a series of bits, representing a number. But each bit
        itself or any sub-group of adjacent bits within the integer may also
        be interpreted as a number.

        Now imagine each bit of an integer as an virtual integer of it's own,
        which may be either $0$ or $1$.
        If we group those bits into pairs, and add up each pair, the result is
        half the number of integers, which may each take the place of the pair
        which was used to calculate them.
        Each of the new integers holds the number of bits set in the original
        pair.
        Now imaging those 2bit-wide integers be grouped in pairs and added up.
        The result will be, again, half the number of integers which are now
        4bit wide.
        The step may be continued until only one number remains, which fills up
        the entire space of the physical integer.
        This integer now holds the number of bits which were set in the original
        integer fed to the algorithm.
        For a visualization of the algorithm, see figure
        \ref{fig:misc-bitcount-hierarchical_merge}).

        \begin{figure}[!h]
            \caption{Hierarchical merges}
            \label{fig:misc-bitcount-hierarchical_merge}
            \begin{center}
                \includegraphics{fig/hierarchical-merge.1}
            \end{center}
        \end{figure}

        The algorithm is in $O(log(n))$ if we manage to perform each step in
        constant time, e.g. independently of the number of pairs being merged.
        This is, indeed, possible using bitwise operations which affect all the
        pairs in parallel.

        To merge each pair, we have to shift the entire physical integer to the
        right by the width of one virtual integer.
        This results in the ``higher'' of the integers in a pair to be on the
        same ``level'' as the ``lower'' one was before.
        If we mask both the result of the shifting operation and the original
        value, we have two clean sets of virtual integers.
        If we add those physical integers the regular way, the overflows of each
        pair will end up in the place where the ``higher'' of the virtual
        integers was before (see fig. \ref{fig:misc-bitcount-single_merge}).
        Since the overflow may be of one bit at most, this is acceptable.

        \begin{figure}[!h]
            \caption{A single merge}
            \label{fig:misc-bitcount-single_merge}
            \begin{center}
                \includegraphics{fig/single-merge.1}
            \end{center}
        \end{figure}

