\section{Bloom filter usage}
\label{sec:bloom_filter}

    %TODO: reference
    A bloom filter is a probabilistic data structure which allows a fast check
    for a potential membership of an item in a set $B$.
    As a data structure, it consists of an array of a constant number $m$ of
    bits $b_0$ to $b_{m-1}$.
    For an empty set, all those bits are set to $0$.
    On insertion of an element into the set, $k$ hash functions $h_0$ to
    $h_{k-1}$ are applied to the element.
    The resulting values are transformed to positions within the array, e.g.
    through the modulo operation.
    Those bits are then set to $1$.
    Given a set of $n$ elements, the membership of an element $e$ may be checked
    by checking whether each bit ``pointed'' to by the hash functions applied to
    the new element are set.
    If one of those bits is not set, the element is not in the corresponding set
    $B$.

    \begin{equation}
        \exists_i : \neg b_{h_i(e)\%m} \Rightarrow e \notin B
    \end{equation}

    If all the bits relevant for the element are set, the element \emph{may} be
    in the set $B$:

    \begin{equation}
        \forall_i : b_{h_i(e)\%m} \Rightarrow
        p(e \in B) = \left(1 - \left(1 - \frac{1}{m} \right)^{kn} \right)^k
    \end{equation}

    In libreset, bloom filters will be used as filters to speed up some
    operations.
    The use of bloom filters will, however, exceed the classical use for
    speeding up lookups.
    The filter may also be used to speed up merges and intersections by allowing
    early recognition of disjunctive sets.

    \subsection{Disjunctive sets}
    \label{sec:bloom_filter-disjunctive_sets}

        Checking whether the intersection of two sets $A$ and $B$ is not empty
        is equivalent to checking whether at least one element of set $A$ is in
        set $B$.
        The check for membership of one element described earlier may also be
        expressed in another way: instead of checking whether one of the bits
        at the positions given by the hashes of the new element is \emph{not}
        set, one could check whether \emph{less then $k$} of those bits are set:

        \begin{equation}
            \left( \sum_i b_{h_i(e)\%m} \right) < k \Rightarrow e \notin B
            \label{eq:bloom_filter-disjunctive_sets-less_than_k}
        \end{equation}

        We may also represent a single item $e$ by the bloom filter $s(e)$ of a
        set which contains only that single item:

        \begin{equation}
            \exists_i : h_i(e)\%m = j \quad\equiv\quad s(e)_j = 1
        \end{equation}

        Using this filter, \ref{eq:bloom_filter-disjunctive_sets-less_than_k}
        may be written as:

        \begin{equation}
            \left( \sum_i b_i \otimes s(e)_i \right) < k \Rightarrow e \notin B
        \end{equation}

        Note that $i$ now represents a bit rather than a hash function.

        Now consider a set $A$, which contains an arbitrary amount of elements.
        If A contains no elements which are in B, the above must be true for
        each of them:

        \begin{equation}
            \forall_{e \in A} \left( \sum_i b_i \otimes s(e)_i \right) < k
            \Rightarrow A \cap B = \emptyset
        \end{equation}

        An even stronger condition for $A$ and $B$ being exclusively disjunctive
        is achieved by pulling the quantor into the term under the sum:

        \begin{equation}
            \left( \sum_i b_i \otimes \bigoplus_{e \in A} s(e)_i \right) < k
            \Rightarrow A \cap B = \emptyset
        \end{equation}

        Now ``OR''-ing all the bloom filters for a sets element together bitwise
        will result in the set's bloom filter itself:

        \begin{equation}
            \bigoplus_{e \in A} s(e)_i = a_i
        \end{equation}

        Thus:

        \begin{equation}
            \left( \sum_i b_i \otimes a_i \right) < k
            \Rightarrow A \cap B = \emptyset
        \end{equation}

        Assume the result of a bitwise conjunction\footnote{``AND''} of the
        bloom filters $a$ and $b$ of two sets be given. If the number of bits
        set in the result is less than the number of hash functions used, the
        sets are exclusively disjunctive.


