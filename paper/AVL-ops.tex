\section{AVL operations and algorithms}
\label{sec:AVL_ops}

    This section will discuss special operations on AVL trees. Basic operations
    like insertion, search and deletion will not be discussed in this document.
    In the following subsections, $n$ will denote the number of elements in the
    AVL.
    Furthermore, for each node a sufficiently small number of sub-elements will
    be assumed for the actual number to be irrelevant for the algorithms'
    classes of complexity.

    \subsection{Isolating subtrees}
    \label{sec:AVL_ops-isolate}

        As previously discussed, buckets represent ranges of integral numbers.
        %TODO: ref
        Since buckets have to be split during resizes and other operations,
        sometimes subtrees have to be split into an AVL with hashes which are
        greater than or equal a specific number and an AVL containing elements
        with hashes lower than that number:
        \begin{equation}
            [a,b) \mapsto [a,c), [c,b)
        \end{equation}

        Since each node's left children's keys are lower than the current's
        node's one while the right children's keys are greater, splitting an
        AVL into items with lower and greater keys is a simple procedure:
        \begin{itemize}
            \item find the first item which does not fall in the category of
                the root node
            \item split it from it's parent
        \end{itemize}
        This results in two AVLs (from which one may not be balanced).

    \subsection{Merging AVLs}
    \label{sec:AVL_ops-merge}

        Similar to isolation, AVLs may also be merged.
        However, merging will not be constrained to AVLs representing
        exclusively disjunct sets, neither regarding the actual items
        nor the ranges they represent.

        \begin{itemize}
            \item split the tree to merge in nodes with keys greater and lower
                than the current AVL's root node's key, extracting the node
                with an \emph{identical} key. ($O(log(n))$)
            \item merge that node with the root node, if present ($O(1)$)
            \item proceed with the corresponding sub-trees
            \item if one of the target's subtree is empty, link the remaining
                subtrees from the source
        \end{itemize}

        This algorithm is in $O(n)$ (every node is processed once).
        For AVLs which represent exclusively disjunct ranges, this algorithm
        is in $O(log(n))$.


    \subsection{Intersecting AVLs}
    \label{sec:AVL_ops-intersect}

        Determination of an AVL's intersection with another AVL is similar to
        merging an AVL into another:

        \begin{itemize}
            \item split the tree to merge in nodes with keys greater and lower
                than the current AVL's root node's key, extracting the node
                with an \emph{identical} key. ($O(log(n))$)
            \item remove all items from that node which are not in the root node
                of the previously isolated subtree ($O(n)$)
            \item proceed with the corresponding sub-trees
        \end{itemize}

        This algorithm is in $O(n)$ (every node is processed once).
        For AVLs which represent exclusively disjunct ranges, this algorithm
        is in $O(log(n))$.


    \subsection{Excluding one AVL's elements from another}
    \label{sec:AVL_ops-exclude}

        Exclusion of one AVL's elements from another AVL is also similar to
        merging AVLs:

        \begin{itemize}
            \item split the tree to merge in nodes with keys greater and lower
                than the current AVL's root node's key, extracting the node
                with an \emph{identical} key. ($O(log(n))$)
            \item remove all items from that node which are in the root node
                of the previously isolated subtree ($O(n)$)
            \item proceed with the corresponding sub-trees
        \end{itemize}

        This algorithm is in $O(n)$ (every node is processed once).
        For AVLs which represent exclusively disjunct ranges, this algorithm
        is in $O(log(n))$


    \subsection{Rebalancing an AVL}
    \label{sec:AVL_ops-rebalance}

        An AVL tree is rebalanced by means of rotations.
        However, if the tree is unbalanced by more than one element, the
        rebalancing is not as trivial as it usually would be for AVL trees.

        By performing rotations, one can move elements from one subtree to
        another.
        For unbalanced trees, however, one has to ensure that, by doing so, the
        imbalance is reduced rather than extended.
        This is achieved by observing the number of elements in each affected
        subtree and basing the choice of rotations to be performed on that
        information.

        The number of elements $n$ which may fit into a subtree, as a function
        of the height $h$, is:

        \begin{equation}
            n(h) = 2^h-1
        \end{equation}

        If the height is increased by $1$, this yields:

        \begin{equation}
            n(h+1)
                = 2^{h+1}-1
                = 2\cdot 2^h -1
                = 2\cdot \left(2^h-1\right) +1
                = 2n(h)+1
        \end{equation}

        Given two subtrees $a$ and $b$ of a node, the height of those nodes
        must not differ by more than one.
        However, the subtrees themselves might not be balanced.
        Because of this, we may base the condition on the number of elements
        rather than the maximal height of a subtree.

        A subtree $a$ being higher than $b$ by more than one may be expressed
        be the following relation:

        \begin{equation}
            n_a > 2n_b + 1
        \end{equation}

        Now consider the following right-rotation with the sub-trees $a$, $b$
        and $c$:

        \begin{figure}[!h]
            \caption{Right rotation}
            \label{fig:AVL_ops-rebalance-rot}
            \begin{center}
                \includegraphics{fig/right-rot.1}
            \end{center}
        \end{figure}

        The rotation has to be carried out, if

        \begin{equation}
            n_a + n_b + 1 = n_l > 2n_c +1
            \label{sec:AVL_ops-rebalance-rot_preq}
        \end{equation}

        Which may also be expressed as either

        \begin{equation}
            n_a + n_b > 2n_c \qquad \mathrm{or} \qquad n_b > 2n_c - n_a
        \end{equation}

        However, the result must not be imbalanced in the opposite direction,
        that is the right subtree's potential height being more than one higher
        than the left subtree's one:

        \begin{equation}
            2n_a + 1 \geq n_r = n_b + n_c + 1
            \label{sec:AVL_ops-rebalance-rot_cond1}
        \end{equation}

        Which may also be expressed as either

        \begin{equation}
            2n_a \geq n_b + n_c \qquad \mathrm{or} \qquad 2n_a - n_c \geq n_b
        \end{equation}

        Combining those two constrains yields:

        \begin{equation}
            2n_a - n_c \geq n_b > 2n_c -n_a
        \end{equation}

        Which may be simplified to:

        \begin{equation}
            n_a > n_c
            \label{sec:AVL_ops-rebalance-rot_cond2}
        \end{equation}

        If the tree is imbalanced according to equation
        \ref{sec:AVL_ops-rebalance-rot_preq} but the condition in equation
        \ref{sec:AVL_ops-rebalance-rot_cond2} is not met, left-rotations have to
        be carried out in the left subtree until the condition is met and the
        right-rotation can be performed.

        After the tree is balanced on the top level, the subtrees may be
        rebalanced recursively.

        The algorithm is expected to be in $O(n)$ for most cases and in
        $O(log(n))$, if only one leaf causes the imbalance.

        %TODO: optional approach


