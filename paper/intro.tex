\section{Introduction}

    %TODO: definition of "set"

    %TODO: refs to some standard source on data structures
    ``Sets'' or ``Maps'' are often implemented using either hash tables or AVL
    trees.  For the former one, insertion, deletion and query of an element via
    a key are generally in $O(1)$ in the average case, but may be in $O(n)$ in
    the worst case.  For AVLs, these operations are in $O(ln(n))$ in the average
    \emph{and} in the worst case. If the usage of a set is restricted to these
    operations, both implementations are suitable.

    However, for many interesting set operations, the behaviour of the
    data-structures is sub-optimal. E.g. determining whether a set is a subset
    of another is an especially slow operation if the set type is implemented on
    top of a hash-table, as for every element of the presumed subset the
    presence in the second set has to be checked. For AVLs, calculating an
    intersection may be done by iterating over both AVLs, using a merge-like
    algorithm to fetch the elements of the intersection in $O(n)$. However, each
    of the elements has to be inserted in the resulting set, making this a
    $O(n\cdot ln(n))$ operation.

    This paper introduces a few ideas and techniques on how these two data-types
    may be combined and on how operations might be implemented to achieve better
    performance than would normally be possible using standard set
    implementations.

